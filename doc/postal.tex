\documentclass[12pt]{article}
\usepackage[
    pdftitle={Postal},
    pdfauthor={Christian Hergert},
    pdfsubject={Postal},
    pdfkeywords={Programming},
    bookmarks, bookmarksopen,
    pdfstartview=FitH,
    colorlinks,linkcolor=blue,citecolor=blue,
    urlcolor=red,
]{hyperref}
\usepackage{color}
\usepackage{listings}
\usepackage{parskip}
\usepackage{graphicx}
\usepackage{tikz}
\usepackage[light,math]{iwona}
\usepackage[T1]{fontenc}
\parskip 7.2pt

\newcommand{\versal}[1]{\noindent{\Huge #1\kern-.10em}}
\newcommand{\file}[1]{{\bf\ttfamily #1}}
\newcommand{\ident}[1]{{\it\ttfamily #1}}
\newcommand{\shell}[1]{{\it\ttfamily #1}}
\newcommand{\python}[1]{{\it\ttfamily #1}}
\newcommand{\ruby}[1]{{\it\ttfamily #1}}
\newcommand{\book}[2]{{\it\ttfamily #1} by {\it #2}}
\newcommand{\program}[1]{\it\ttfamily #1}

\lstnewenvironment{code}[1]
{
    \lstset{language=C,
            showstringspaces=false,
            basicstyle=\ttfamily\small,
            caption=#1,
            numbers=left}
}
{}


\lstnewenvironment{Terminal}
{
    \lstset{basicstyle=\ttfamily,
            breaklines=true}
}
{}


\title{Postal \\ Push Notification Server}
\author{Christian Hergert \\
    \small \texttt{christian@catch.com}
}

\date{November 2012}

\begin{document}
\maketitle

Postal is a free and open source push notification server written in C.

\section{Introduction}

\section{REST API}

\subsection{Creating and Managing Devices}

\subsubsection{Adding a Device}

\subsubsection{Removing a Device}

\subsection{Sending Notifications}

\section{Metrics}

Many system operators will want to know what is going on with their system.
Postal provides a mechanism for being notified about events happening within the system.
This mechanism uses a \verb|PUBSUB| channel via \verb|Redis| to deliver JSON formatted events.
This feature requires that you configure Postal with \verb|--enable-redis=yes|.

\subsection{Configuration}

Metrics delivery via \verb|Redis| provides the following configuration options.

\begin{figure}[h!]
\centering
\begin{tabular}{l l l}
\hline
Option & Value & Description \\
\hline
enabled & \verb|true| or \verb|false| & If metric delivery is enabled. \\
host & \emph{hostname} & The hostname of the Redis server. \\
port & \verb|6379| & The port of the Redis server. \\
channel & \emph{event:postal} & The Redis PUBSUB channel name. \\
\hline
\end{tabular}
\caption{Redis Configuration Options}
\end{figure}

The options should be stored in the \verb|redis| group in \file{postal.conf}.
See the example below.

\begin{figure}[h!]
\begin{Terminal}
[redis]
enabled = true
host = localhost
port = 6379
channel = events:postal
\end{Terminal}
\caption{Redis Configuration Example}
\end{figure}

\subsection{Events}

The following sections describe the various types of events that you may receive.

\subsubsection{Device Added}

This event is published when Postal has received a request to add a new device.

\begin{Terminal}
{
  "Action": "device-added",
  "DeviceType": "gcm",
  "DeviceToken": "12341234-12341234",
  "User": "000000000000111122223333"
}
\end{Terminal}

\subsubsection{Device Updated}

This event is published when Postal has updated the information for a device.
This could happen if an API request has been processed to change the attributes of the device.

\begin{Terminal}
{
  "Action": "device-updated",
  "DeviceType": "gcm",
  "DeviceToken": "12341234-12341234",
  "User": "000000000000111122223333"
}
\end{Terminal}

\subsubsection{Device Removed}

This event is published when Postal has been requested to remove a device.
This could happen from an API request that specifically requests the removal of a device.
This could also happen if push notifications have been disabled for a device.
Additionally, with APS, this could happen if the feedback service has requested the removal of a device.

\begin{Terminal}
{
  "Action": "device-removed",
  "DeviceType": "gcm",
  "DeviceToken": "12341234-12341234",
  "User": "000000000000111122223333"
}
\end{Terminal}

\subsubsection{Device Notified}

This event is published when Postal has successfully delivered a notification to the gateway.
This does not mean that the device has received the notification.
The various gateways to not provide this information so that is not available.

\begin{Terminal}
{
  "Action": "device-notified",
  "DeviceType": "gcm",
  "DeviceToken": "12341234-12341234",
  "User": "000000000000111122223333"
}
\end{Terminal}

\end{document}
